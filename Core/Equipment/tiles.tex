% REVIEW: Write section
% TODO: Refactor to Resources and Diagrams
\section{Tiles}\label{core:sec:tiles}

The game is played with 136 tiles. Each tile has exactly four copies of it in the deck, and no further.

\subsection{Numbered Tiles}\label{core:ssec:num-tiles}

There are three suits of numbered tiles. These tiles may form runs. They are ordered 1 through 9 in each of the following examples:

\begin{center}
\textbf{\large Pinzu -- \ruby{筒子}{ピンズ}} \textit{``Dots''}

{\mahjong[1.5cm]{123456789p}}
\end{center}

\begin{center}
\textbf{\large Souzu -- \ruby{索子}{ソーズ}} \textit{``Sticks''}

{\mahjong[1.5cm]{123456789s}}
\end{center}

\begin{center}
\textbf{\large Manzu -- \ruby{萬子}{マンズ}} \textit{``Mans''}

{\mahjong[1.5cm]{123456789m}}
\end{center}

\subsection{Honour Tiles}\label{core:ssec:hon-tiles}

% TODO: Better Tile Names
There are two suits of honour tiles. These tiles may \emph{not} form runs. Each of the honours do not follow a simple numbered structure and will be named below in each of their sections

\begin{center}
\textbf{\large Kazepai -- \ruby{風牌}{かぜぱい}} \textit{``Winds''}

{\mahjong[1.5cm]{1234z}}

\parbox{1cm}{East} \parbox{1cm}{South} \parbox{1cm}{West} \parbox{1cm}{North}
\end{center}

\begin{center}
\textbf{\large Sangenpai -- \ruby{三元牌}{さんげんぱい}} \textit{``Dragons''}

{\mahjong[1.5cm]{567z}}

\parbox{1cm}{White} \parbox{1cm}{Green} \parbox{1cm}{Red}
\end{center}

\subsection{Red Dora}\label{core:ssec:akadora}

Some tiles can be recoloured to be entirely Red. These tiles are known as Red or Aka Dora. Each Red Dora must replace a non-red couterpart. This ruleset uses one of each red five shown below. This then means that there are three normal fives, and one red five per suit.

\begin{center}
{\mahjong[1.5cm]{0p0s0m}}
\end{center}

% REVIEW: Write Section
\section{Scoring}\label{core:sec:scoring}

\subsection{Paying out a Winning Hand}\label{core:sec:paying-hand}

In order to pay out a winning hand, follow the following procedure:
\begin{enumerate}
	\item Count the Han earned from Yaku~\ref{core:sec:yaku}, and Bonuses~\ref{core:ssec:1-han-bonus}.
	\item If that number is 4 or lower, count the Fu~\ref{core:sec:fu} the hand earns. Round that to the next 10, with the exception of hands winning with Chiitoitsu~\pr{yaku}{Chiitoitsu}, that hand is always worth 25 fu.
\end{enumerate}

After determining the Han, and if needed Fu, consult the table in Figure~\ref{core:fig:points} to determine the score. Taking into account Dealer or Non-dealer, Ron or Tsumo. Note that the third column is what is paid by all for a dealer Tsumo.

\clearpage
\begin{figure}[h]\centering
\begin{tabular}{|H|H|H|H||S|S|S|S|}
\multicolumn{4}{c}{Han}   & \multicolumn{2}{c}{Non-dealer}           & \multicolumn{2}{c}{Dealer} \\\hline
1 & 2 & 3 & 4          & Ron    & Tsumo &  Tsumo &    Ron \\\hline\hline
30  &     &    &    &  1,000 &   300 &    500 &  1,500 \\\hline
40  & 20  &    &    &  1,300 &   400 &    700 &  2,000 \\\hline
50  & 25  &    &    &  1,600 &   400 &    800 &  2,400 \\\hline
60  & 30  &    &    &  2,000 &   500 &  1,000 &  2,900 \\\hline
70  &     &    &    &  2,300 &   600 &  1,200 &  3,400 \\\hline
80  & 40  & 20 &    &  2,600 &   700 &  1,300 &  3,900 \\\hline
90  &     &    &    &  2,900 &   800 &  1,500 &  4,400 \\\hline
100 & 50  & 25 &    &  3,200 &   800 &  1,600 &  4,800 \\\hline
110 &     &    &    &  3,600 &   900 &  1,800 &  5,300 \\\hline
    & 60  & 30 &    &  3,900 & 1,000 &  2,000 &  5,800 \\\hline
    & 70  &    &    &  4,500 & 1,200 &  2,300 &  6,800 \\\hline
    & 80  & 40 & 20 &  5,200 & 1,300 &  2,600 &  7,700 \\\hline
    & 90  &    &    &  5,800 & 1,500 &  2,900 &  8,700 \\\hline
    & 100 & 50 & 25 &  6,400 & 1,600 &  3,200 &  9,600 \\\hline
    & 110 &    &    &  7,100 & 1,800 &  3,600 & 10,600 \\\hline
    &     & 60 & 30 &  7,700 & 2,000 &  3,900 & 11,600 \\\hline
\multicolumn{2}{|c|}{Mangan 5} & + & + &  8,000 & 2,000 &  4,000 & 12,000 \\\hline
\multicolumn{4}{|c|}{Haneman 6,7}      & 12,000 & 3,000 &  6,000 & 18,000 \\\hline
\multicolumn{4}{|c|}{Baiman 8,9,10}    & 16,000 & 4,000 &  8,000 & 24,000 \\\hline
\multicolumn{4}{|c|}{Sanbaiman 11,12}  & 24,000 & 6,000 & 12,000 & 36,000 \\\hline
\multicolumn{4}{|c|}{Yakuman 13}       & 32,000 & 8,000 & 16,000 & 48,000 \\\hline
\end{tabular}

\begin{tabular}{|r||F|}
\multicolumn{2}{c}{{\footnotesize Base Fu}} \\\hline
Winning & 20 \\ \hline
Closed Ron & +10 \\ \hline
Pinfu Tsumo & =20 \\ \hline
7 Pairs & =25 \\ \hline
\end{tabular}
\hspace{6pt}
\begin{tabular}{|c|r||F|F|}
\multicolumn{2}{c}{{\footnotesize Groups}} & \multicolumn{1}{c}{{\scriptsize Open}} & \multicolumn{1}{c}{{\scriptsize Closed}} \\\hline
\multirow{2}{*}{\rotatebox{90}{{\scriptsize Triplet}}} & Simple & 2 & 4 \\ \cline{2-4}
& Orphan & 4 & 8 \\ \hline
\multirow{2}{*}{\rotatebox[origin=c]{90}{{\footnotesize Quad}}} & Simple & 8 & 16 \\ \cline{2-4}
& Orphan & 16 & 32 \\ \hline
\end{tabular}
\hspace{6pt}
\begin{tabular}{|r||F|}
\multicolumn{2}{c}{{\footnotesize Wait/Pair Fu}} \\\hline
Single Wait & 2 \\ \hline
Yakuhai Pair & 2 \\ \hline
Tsumo & 2 \\ \hline
\end{tabular}

\caption{In-Game Scoring Table}\label{core:fig:points}
\end{figure}
\clearpage

\subsection{End of Game Scores}\label{core:sec:end-scores}

At the end of the game, you will convert your scores to what will be called Match Points. Then you will apply bonuses and/or penalties based on your placement in the game. This gets the final match score that you will carry with you to your next game in the series.

The procedure for doing it is as follows:
\begin{enumerate}
	\item Subtract your ending score from 30,000. This 30,000 is called the Return Score.
	\item Then divide the resulting value by 1,000.
	\item Then apply the Placement Bonus, also called Uma. In this case, it is \(+30/+10/-10/-30\).
	\item Then apply the First Bonus, also called Oka. It is calculated as \(\frac{4\times(\text{Return Score} - \text{Start Score})}{1,000}\), in this ruleset that is \(+20\). As the name suggests, this applies only to first place.
\end{enumerate}

To give an example of the above procedure:
\begin{itemize}
	\item A game ended with the scoreline of: \\ \(41,300,\; 24,900,\; 21,800,\; 12,000\)
	\item After paying back the return score it becomes: \\ \(11,300,\; \blacktriangle 5,100,\; \blacktriangle 8,200,\; \blacktriangle 18,000\)
	\item Dividing by 1,000 becomes: \\ \(11.3, \; \blacktriangle 5.1, \; \blacktriangle 8.2, \; \blacktriangle 18.0\) 
	\item Adding the Uma: \\ \(41.3, \; 4.9, \; \blacktriangle 18.2, \; \blacktriangle 48.0\) 
	\item Then the Oka: \\ \(61.3, \; 4.9, \; \blacktriangle 18.2, \; \blacktriangle 48.0\) 
\end{itemize}
% REVIEW: Write Section
% REVIEW: Make Diagrams
\section{Setting up a Hand}\label{core:sec:hand}

\subsection{Shuffling the Tiles}\label{core:ssec:shuffling}

Before the hand can begin, the tiles must \emph{all} be flipped face down then shuffled thoroughly. This is done by everyone pushing the tiles into one another in a collaborative way. It is recommended to keep your palms on the mat, and push tiles inwards to the center to mimic the pattern of swirling water down the drain. While doing so ensure that a minimal amount of tiles get flipped up, and should a tile get flipped up it is flipped down quickly. There is no need to push with a lot of force. It is recommended to shuffle for about thirty seconds.

Then players will grab tiles from the shuffled pool and build two rows of seventeen tiles. Should a tile be accidentally flipped face up during this process, flip it face down and push it back into the shuffled tiles. You should not be building your own wall with tiles that you accidentally flipped face up. Once both rows are completed, stack the row closer to you on top of the row further away from you. Then push your part of the wall closer to the middle to make a sort of square in the middle.

\subsection{Breaking the Wall}\label{core:ssec:breaking}

Once the wall has been built, pass the two dice to the dealer. They will then roll them and add them together to determine where to break the wall. This is analogous to cutting the deck for games using cards. To determine where exactly to break the wall, starting from your wall, count counterclockwise wall by wall until you hit the number rolled. Then count a number of stacks of tiles clockwise in that wall equal to the number rolled. That is where the break in the wall will be.

Below is an example diagram showing where the wall break is on a roll of seven. You will count seven walls, getting to the West player's wall, then break that wall between the seventh and eighth stack going clockwise over it.

\begin{figure}[H]
\begin{center}
\begin{tikzpicture}[scale=0.25]\centering
	% Walls
	\foreach \y in {0,-18.4}{
		\foreach \x in {0,1,...,16}{
			% horizontal walls
			\draw (\x,\y) rectangle (\x+1,\y+1.4);
			\fill [pattern=north east lines] (\x,\y) rectangle (\x+1,\y+1.4);
			% vertical walls
			\draw (-\y,-\x) rectangle (-\y-1.4,-\x-1);
			\fill [pattern=north east lines] (-\y,-\x) rectangle (-\y-1.4,-\x-1);
		}
	}
	% Numbers
	\draw (8.5, -17) node [above] {$5, 9$};
	\draw (17, -8.5) node [left] {$2, 6, 10$};
	\draw (8.5, 0) node [below] {$3, 7, 11$};
	\draw (0, -8.5) node [right] {$4, 8, 12$};
	% Arrows
	\draw [thick, ->] (8.5,-10.5) arc (270:585:2);


	% SPLIT OF WALLS
	\draw [->] (17.5, 0.5) -- (18.8, 1.8) -- node [midway, above] {$7$} (28.5,1.8) -- (28.5, 0.9);


	% vertical walls
	\foreach \y in {-0,-1,...,-16}{
		\foreach \x in {21,39.4}{
			\draw (\x-1.4,\y-1) rectangle (\x,\y);
			\fill [pattern=north east lines] (\x-1.4,\y-1) rectangle (\x,\y);
		}
	}
	% dealer wall
	\foreach \x in {21, 22,...,37}{
		\draw (\x+1,-18.4) rectangle (\x,-17);
		\fill [pattern=north east lines] (\x+1,-18.4) rectangle (\x,-17);
	}
	% cut wall
	\foreach \x in {21, 22,...,27, 29, 30,...,38}{
		\draw (\x+1,+1.4) rectangle (\x,0);
		\fill [pattern=north east lines] (\x+1,+1.4) rectangle (\x,0);
	}

	% % NW Corner
	% \draw [red] (20,0) node {x};
	% \draw [blue] (20,0) rectangle (21, 1.4);
	% \draw [blue] (18.6,-1) rectangle (20,0);
	% % NE Corner
	% \draw [red] (37,0) node {x};
	% \draw [blue] (36,0) rectangle (37, 1.4);
	% \draw [blue] (38.4,-1) rectangle (37,0);
	% % SW Corner
	% \draw [red] (20,-17) node {x};
	% \draw [blue] (20,-18.6) rectangle (21, -17);
	% \draw [blue] (18.6,-16) rectangle (20,-17);
	% % SE Corner
	% \draw [red] (37,-17) node {x};
	% \draw [blue] (36,-18.4) rectangle (37, -17);
	% \draw [blue] (38.4,-16) rectangle (37,-17);
\end{tikzpicture}
\caption{A diagram showing how to break the wall properly.}
\end{center}
\end{figure}


\subsection{Distributing the Tiles}\label{core:ssec:distributing}

After the wall is broken, everyone will draw their own tiles to make their own hands. Starting from the dealer, draw the first two stacks of the wall after the break. It will be a sort of \(2 \times 2\) block. Then the next player will draw their block of tiles. This continues until everyone has drawn three blocks. Everyone in turn order will draw their thirteenth tile, then the dealer draws their fourteenth tile. It is common practice for the dealer to draw both their thirteenth and fourteenth tile together by gripping both at once.

Below is a diagram showing how exactly each player will draw their hands on the example roll of a seven. Note that for the section under the *, both tiles labelled with an ``E'' are the ones the dealer would draw if they wish to draw both their thirteenth and fourteenth tile at once.

\begin{figure}[H]
\begin{center}
\begin{tikzpicture}[scale=0.25]\centering
	% vertical walls
	\foreach \y in {-0,-1,...,-16}{
		\foreach \x in {0,18.4}{
			\draw (\x-1.4,\y-1) rectangle (\x,\y);
			\fill [pattern=north east lines] (\x-1.4,\y-1) rectangle (\x,\y);
		}
	}
	% dealer wall
	\foreach \x in {0, 1,..., 16}{
		\draw (\x+1,-18.4) rectangle (\x,-17);
		\fill [pattern=north east lines] (\x+1,-18.4) rectangle (\x,-17);
	}
	% cut wall
	\foreach \x in {0, 1,..., 6, 8, 9,..., 17}{
		\draw (\x+1,+1.4) rectangle (\x,0);
		\fill [pattern=north east lines] (\x+1,+1.4) rectangle (\x,0);
	}
	% Marking who draws what
	\draw [|-|, shorten >=1pt] (8, 2) -- node [midway, above] {E} (10, 2);
	\draw [|-|, shorten >=1pt] (10, 2) -- node [midway, above] {S} (12, 2);
	\draw [|-|, shorten >=1pt] (12, 2) -- node [midway, above] {W} (14, 2);
	\draw [|-|, shorten >=1pt] (14, 2) -- node [midway, above] {N} (16, 2);
	\draw [|-|, shorten >=1pt] (16, 2) -- node [midway, above] {E} (18, 2);
	\draw [|-|, shorten >=1pt] (19, 0) -- node [midway, right] {S} (19, -2);
	\draw [|-|, shorten >=1pt] (19, -2) -- node [midway, right] {W} (19, -4);
	\draw [|-|, shorten >=1pt] (19, -4) -- node [midway, right] {N} (19, -6);
	\draw [|-|, shorten >=1pt] (19, -6) -- node [midway, right] {E} (19, -8);
	\draw [|-|, shorten >=1pt] (19, -8) -- node [midway, right] {S} (19, -10);
	\draw [|-|, shorten >=1pt] (19, -10) -- node [midway, right] {W} (19, -12);
	\draw [|-|, shorten >=1pt] (19, -12) -- node [midway, right] {N} (19, -14);
	\draw [|-|, shorten >=1pt] (19, -14) -- node [midway, right] {$*$} (19, -17);

	% SPLIT TO SIDE VIEWS
	% side-view block draw
	\foreach \x in {23,24.5, 27,28.5, 31,32.5, 35,36.5, 39,40.5}{
		\foreach \y in {0, -1}{
			\draw (\x,\y) rectangle (\x+1.5,\y-1);
			\fill [pattern=north east lines] (\x,\y) rectangle (\x+1.5,\y-0.3);
			\draw (\x,\y-0.3) -- (\x+1.5,\y-0.3);
		}
	}

	\foreach \x in {23, 27, 31, 35, 39}{
		\draw [dashed] (\x-0.3, 0.3) rectangle (\x+3.3, -2.3);
	}
	\draw (24.5, 0.3) node [above] {E};
	\draw (28.5, 0.3) node [above] {S};
	\draw (32.5, 0.3) node [above] {W};
	\draw (36.5, 0.3) node [above] {N};
	\draw (40.5, 0.3) node [above] {E};

	\draw [|-|] (7.7, 4) -- (18.3, 4);
	\draw [->] (13.0, 4) -- (13,5) -- (32,5) -- (32,2);
	\draw [|-|] (22.7, 2) -- (42.3, 2);

	% side-view jump-draw
	\foreach \x in {28, 31, 34}{
		\foreach \y in {-9, -11}{
			\draw (\x,\y) rectangle (\x+3,\y-2);
			\fill [pattern=north east lines] (\x,\y) rectangle (\x+3,\y-0.7);
			\draw (\x,\y-0.7) -- (\x+3,\y-0.7);
		}
	}
	\draw (29.5, -10) node [circle, fill=white, inner sep=1pt] {\small E};
	\draw (29.5, -12) node [circle, fill=white, inner sep=1pt] {\small S};
	\draw (32.5, -10) node [circle, fill=white, inner sep=0pt] {\small W};
	\draw (32.5, -12) node [circle, fill=white, inner sep=1pt] {\small N};
	\draw (35.5, -10) node [circle, fill=white, inner sep=1pt] {\small E};

	\draw [->] (21, -15.5) -- (32.5,-15.5) -- (32.5,-14);
	\draw [|-|] (27.7, -14) -- (37.3, -14);

\end{tikzpicture}
\caption{A diagram showing how to draw your opening hands correctly.}
\end{center}
\end{figure}


\subsection{Setting up the Dead Wall}\label{core:ssec:dead-wall}

While this is all going on, the player whose wall was broken should prepare the dead wall. In order to do so, they must drop down the Rinshan Tile, and reveal the first Dora indicator. The below diagram shows you how to do so. The Dora indicator is the top tile of the third stack in, counting counterclockwise from the break, and the Rinshan Tile is the top tile in the first stack in. We drop the Rinshan Tile to ensure it doesn't accidentally get knocked over during the game. This tile is very important, because it will be the tile that is drawn after calling a Kan \pr{call}{Kan}. In addition to that, winning by drawing it after a Kan gives you the Rinshan Kaihou yaku \pr{yaku}{Rinshan Kaihou}. So we want to make sure that that tile is not accidentally exposed.

\begin{figure}[H]
\begin{center}
\begin{tikzpicture}[scale=0.25]
	% east wall
	\foreach \x in {0, 1,..., 16}{
		\draw (\x+1,-18.4) rectangle (\x,-17);
		\fill [pattern=north east lines] (\x+1,-18.4) rectangle (\x,-17);
	}
	% south wall
	\foreach \y in {-16}{
		\draw (17,\y-1) rectangle (18.4,\y);
		\fill [pattern=north east lines] (17,\y-1) rectangle (18.4,\y);
	}
	% west wall
	\foreach \x in {0, 1,..., 6}{
		\draw (\x+1,+1.4) rectangle (\x,0);
		\fill [pattern=north east lines] (\x+1,+1.4) rectangle (\x,0);
	}
	% north wall
	\foreach \y in {-0,-1,...,-16}{
		\draw (-1.4,\y-1) rectangle (0,\y);
		\fill [pattern=north east lines] (-1.4,\y-1) rectangle (0,\y);
	}
	% bottom layer
	\foreach \x in {4,6,...,18}{
		\draw (\x,-8.5) rectangle (\x+2,-10);
		\fill [pattern=north east lines] (\x,-8.5) rectangle (\x+2,-9);
		\draw (\x,-9) rectangle (\x+2,-9);
	}
	% top layer
	\foreach \x in {4,6,...,10,14}{
		\draw (\x,-7) rectangle (\x+2,-8.5);
		\fill [pattern=north east lines] (\x,-7) rectangle (\x+2,-7.5);
		\draw (\x,-7.5) rectangle (\x+2,-7.5);
	}
	% dora
	\draw (12,-7) rectangle (14,-8.5);
	\fill [pattern=north east lines] (12,-8) rectangle (14,-8.5);
	\draw (12,-8) -- (14,-8);
	% flipping the dora
	\draw [->,shorten >=1pt] (12.5,-7.75) |- (13,-6) -| (13.5,-7);
	% % dropping the tile
	\draw [dashed, thin] (16.1,-7.1) rectangle (17.9,-8.4);
	\draw [->,shorten >=1pt] (17,-7.75) -| (19,-8.5);

	% breakout lines
	\draw [|-|] (-0.3, 1.7) -- (7.3, 1.7);
	\draw [->] (3.5, 1.7) -- (3.5,3) -- (12,3) -- (12,-5.5);
	\draw [|-|] (3.7, -5.5) -- (20.3, -5.5);
\end{tikzpicture}
\caption{A diagram showing what needs to be done to properly set up the dead wall.}
\end{center}
\end{figure}

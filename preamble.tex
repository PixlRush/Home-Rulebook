\documentclass{article}

\usepackage[includehead,
			margin=0.5in, 
			paperwidth=5.5in, 
			paperheight=8.5in
		   ]{geometry}
\setlength{\parskip}{6pt}

% Table Controllers
\usepackage{tabulary, multirow, makecell, float}
\setlength{\tabcolsep}{2pt}        % Space between Columns
% Special columns
\newcolumntype{S}{>{\raggedleft\arraybackslash}p{0.5in}}
\newcolumntype{H}{>{\centering\arraybackslash}p{0.25in}}
\newcolumntype{F}{>{\centering\arraybackslash}p{0.25in}}

% Support for Japanese Text
\usepackage{xeCJK, ruby}
\setCJKmainfont{Toppan Bunkyu Mincho}
\newCJKfontfamily\cjkbf{Toppan Bunkyu Midashi Mincho}
\renewcommand{\rubysep}{0ex}

% Support for loading images
\usepackage{graphicx}

% Misc Imports
\usepackage[hidelinks]{hyperref}
\usepackage{wrapfig}
\usepackage{glossaries}
\makeglossaries
\usepackage{mahjong}

% Set up command for Yaku
\def\rulesetprefix{undef}
% Generic command to be wrapped
\newcommand{\ruleentry}[6]{
	\def\e{}\def\t{#4}
	% #1 - Romaji Name
	% #2 - English Name
	% #3 - Japanese Name
	% #4 - Flags for the Yaku
	% #5 - Description of the Yaku
	\subsubsection*{\large #1~--~{\cjkbf #3} {\normalfont\itshape\normalsize``#2''\/}}\label{\rulesetprefix:#6:#1}
	\vspace*{-12pt}
	\ifx\e\t\else \textit{\footnotesize #4\\} \fi
	\hspace*{\parindent}#5
}
% Rule item wrappers
\newcommand{\yaku}[5]{\ruleentry{#1}{#2}{#3}{#4}{#5}{yaku}}
\newcommand{\bonus}[5]{\ruleentry{#1}{#2}{#3}{#4}{#5}{bonus}}
\newcommand{\fu}[5]{\ruleentry{#1}{#2}{#3}{#4}{#5}{fu}}

% Referencer Wrapper
\newcommand{\pr}[2]{\textit{(p\pageref{\rulesetprefix:#1:#2})}}
\newcommand{\prs}[3]{\textit{(p\pageref{\rulesetprefix:#1:#2}/\pageref{\rulesetprefix:#1:#3})}}

% Useful Symbols
\def\symbnegate{\( \times \)}
\def\symbalert{\( !!\; \)}
\def\closedhand{\symbalert Closed Hand Only}
\def\brokenhand{\symbnegate Broken by Chii, Pon, or Kan calls}
\def\morevaluable{\( + \)}
\def\lessvaluable{\( - \)}
\def\upgradesto{\( \leftarrow \)}
\def\upgradesfrom{\( \rightarrow \)}
\def\upgradestoother{\( \Leftarrow \)}
\def\upgradesfromother{\( \Rightarrow \)}

% Set up generic titling
\renewcommand\title{}
\newcommand\subtitle{}
\renewcommand\author{Matthew Rappaport}

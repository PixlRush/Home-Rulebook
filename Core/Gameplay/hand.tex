% REVIEW: Write Section
% TODO: Make Diagrams
\section{Setting up a Hand}\label{core:sec:hand}

\subsection{Breaking the Wall}

At the beginning of every hand, the dealer will roll two six-sided dice. This determines precisely where to break the wall. Firstly, count counterclockwise wall by wall starting from the dealer's wall. Then, upon counting to the value on the dice, break the wall that many tiles from the right of that wall.

For example, on the most common roll of a seven. You will count until you get to the West player's wall. After which you will break it seven from the right.

\begin{figure}[H]
\begin{center}
\begin{tikzpicture}[scale=0.25]\centering
	% Walls
	\foreach \y in {0,-18.4}{
		\foreach \x in {0,1,...,16}{
			% horizontal walls
			\draw (\x,\y) rectangle (\x+1,\y+1.4);
			\fill [pattern=north east lines] (\x,\y) rectangle (\x+1,\y+1.4);
			% vertical walls
			\draw (-\y,-\x) rectangle (-\y-1.4,-\x-1);
			\fill [pattern=north east lines] (-\y,-\x) rectangle (-\y-1.4,-\x-1);
		}
	}
	% Numbers
	\draw (8.5, -17) node [above] {$5, 9$};
	\draw (17, -8.5) node [left] {$2, 6, 10$};
	\draw (8.5, 0) node [below] {$3, 7, 11$};
	\draw (0, -8.5) node [right] {$4, 8, 12$};
	% Arrows
	\draw [thick, ->] (8.5,-10.5) arc (270:585:2);


	% SPLIT OF WALLS
	\draw [->] (17.5, 0.5) -- (18.8, 1.8) -- node [midway, above] {$7$} (28.5,1.8) -- (28.5, 0.9);


	% vertical walls
	\foreach \y in {-0,-1,...,-16}{
		\foreach \x in {21,39.4}{
			\draw (\x-1.4,\y-1) rectangle (\x,\y);
			\fill [pattern=north east lines] (\x-1.4,\y-1) rectangle (\x,\y);
		}
	}
	% dealer wall
	\foreach \x in {21, 22,...,37}{
		\draw (\x+1,-18.4) rectangle (\x,-17);
		\fill [pattern=north east lines] (\x+1,-18.4) rectangle (\x,-17);
	}
	% cut wall
	\foreach \x in {21, 22,...,27, 29, 30,...,38}{
		\draw (\x+1,+1.4) rectangle (\x,0);
		\fill [pattern=north east lines] (\x+1,+1.4) rectangle (\x,0);
	}

	% % NW Corner
	% \draw [red] (20,0) node {x};
	% \draw [blue] (20,0) rectangle (21, 1.4);
	% \draw [blue] (18.6,-1) rectangle (20,0);
	% % NE Corner
	% \draw [red] (37,0) node {x};
	% \draw [blue] (36,0) rectangle (37, 1.4);
	% \draw [blue] (38.4,-1) rectangle (37,0);
	% % SW Corner
	% \draw [red] (20,-17) node {x};
	% \draw [blue] (20,-18.6) rectangle (21, -17);
	% \draw [blue] (18.6,-16) rectangle (20,-17);
	% % SE Corner
	% \draw [red] (37,-17) node {x};
	% \draw [blue] (36,-18.4) rectangle (37, -17);
	% \draw [blue] (38.4,-16) rectangle (37,-17);
\end{tikzpicture}
\caption{A diagram showing how to break the wall properly.}
\end{center}
\end{figure}


\subsection{Distributing the Tiles}

After breaking the wall, it is time to distribute the tiles to the players. Starting from the dealer, they will draw four tiles. Specifically a \(2 \times 2\) block of tiles. After which the player to their right will grab the next \(2 \times 2\) block. This continues until everyone has twelve tiles in their hand. After which, each player in turn order will draw one tile. Lastly, the dealer draws their normal draw to start their turn off.

The dealer may draw their thirteenth and fourteenth tiles together in a sort of jump draw, however that is nothing more than a shortcut for the above procedure.

\begin{figure}[H]
\begin{center}
\begin{tikzpicture}[scale=0.25]\centering
	% vertical walls
	\foreach \y in {-0,-1,...,-16}{
		\foreach \x in {0,18.4}{
			\draw (\x-1.4,\y-1) rectangle (\x,\y);
			\fill [pattern=north east lines] (\x-1.4,\y-1) rectangle (\x,\y);
		}
	}
	% dealer wall
	\foreach \x in {0, 1,..., 16}{
		\draw (\x+1,-18.4) rectangle (\x,-17);
		\fill [pattern=north east lines] (\x+1,-18.4) rectangle (\x,-17);
	}
	% cut wall
	\foreach \x in {0, 1,..., 6, 8, 9,..., 17}{
		\draw (\x+1,+1.4) rectangle (\x,0);
		\fill [pattern=north east lines] (\x+1,+1.4) rectangle (\x,0);
	}
	% Marking who draws what
	\draw [|-|, shorten >=1pt] (8, 2) -- node [midway, above] {E} (10, 2);
	\draw [|-|, shorten >=1pt] (10, 2) -- node [midway, above] {S} (12, 2);
	\draw [|-|, shorten >=1pt] (12, 2) -- node [midway, above] {W} (14, 2);
	\draw [|-|, shorten >=1pt] (14, 2) -- node [midway, above] {N} (16, 2);
	\draw [|-|, shorten >=1pt] (16, 2) -- node [midway, above] {E} (18, 2);
	\draw [|-|, shorten >=1pt] (19, 0) -- node [midway, right] {S} (19, -2);
	\draw [|-|, shorten >=1pt] (19, -2) -- node [midway, right] {W} (19, -4);
	\draw [|-|, shorten >=1pt] (19, -4) -- node [midway, right] {N} (19, -6);
	\draw [|-|, shorten >=1pt] (19, -6) -- node [midway, right] {E} (19, -8);
	\draw [|-|, shorten >=1pt] (19, -8) -- node [midway, right] {S} (19, -10);
	\draw [|-|, shorten >=1pt] (19, -10) -- node [midway, right] {W} (19, -12);
	\draw [|-|, shorten >=1pt] (19, -12) -- node [midway, right] {N} (19, -14);
	\draw [|-|, shorten >=1pt] (19, -14) -- node [midway, right] {$*$} (19, -17);

	% SPLIT TO SIDE VIEWS
	% side-view block draw
	\foreach \x in {23,24.5, 27,28.5, 31,32.5, 35,36.5, 39,40.5}{
		\foreach \y in {0, -1}{
			\draw (\x,\y) rectangle (\x+1.5,\y-1);
			\fill [pattern=north east lines] (\x,\y) rectangle (\x+1.5,\y-0.3);
			\draw (\x,\y-0.3) -- (\x+1.5,\y-0.3);
		}
	}

	\foreach \x in {23, 27, 31, 35, 39}{
		\draw [dashed] (\x-0.3, 0.3) rectangle (\x+3.3, -2.3);
	}
	\draw (24.5, 0.3) node [above] {E};
	\draw (28.5, 0.3) node [above] {S};
	\draw (32.5, 0.3) node [above] {W};
	\draw (36.5, 0.3) node [above] {N};
	\draw (40.5, 0.3) node [above] {E};

	\draw [|-|] (7.7, 4) -- (18.3, 4);
	\draw [->] (13.0, 4) -- (13,5) -- (32,5) -- (32,2);
	\draw [|-|] (22.7, 2) -- (42.3, 2);

	% side-view jump-draw
	\foreach \x in {28, 31, 34}{
		\foreach \y in {-9, -11}{
			\draw (\x,\y) rectangle (\x+3,\y-2);
			\fill [pattern=north east lines] (\x,\y) rectangle (\x+3,\y-0.7);
			\draw (\x,\y-0.7) -- (\x+3,\y-0.7);
		}
	}
	\draw (29.5, -10) node [circle, fill=white, inner sep=1pt] {\small E};
	\draw (29.5, -12) node [circle, fill=white, inner sep=1pt] {\small S};
	\draw (32.5, -10) node [circle, fill=white, inner sep=0pt] {\small W};
	\draw (32.5, -12) node [circle, fill=white, inner sep=1pt] {\small N};
	\draw (35.5, -10) node [circle, fill=white, inner sep=1pt] {\small E};

	\draw [->] (21, -15.5) -- (32.5,-15.5) -- (32.5,-14);
	\draw [|-|] (27.7, -14) -- (37.3, -14);

\end{tikzpicture}
\caption{A diagram showing how to draw your opening hands correctly.}
\end{center}
\end{figure}


\subsection{Setting up the Dead Wall}

After the tiles are distributed, the player whose wall was broken should set up the dead wall. This lies on the other side of the break. Flip the top tile of the third stack in face up. This is the Dora indicator. We deliberately leave four and only four tiles to the left of the Dora indicator for replacement draws for calling Kans \prs{ssec}{calling-open-kan}{calling-closed-kan}. Additionally, place the top tile of the leftmost stack to the left of the bottom tile of the same stack. This is done to avoid it getting knocked over so people will have an easier time earning the Rinshan~Kaihou~yaku~\pr{yaku}{Rinshan Kaihou}.

\begin{figure}[H]
\begin{center}
\begin{tikzpicture}[scale=0.25]
	% east wall
	\foreach \x in {0, 1,..., 16}{
		\draw (\x+1,-18.4) rectangle (\x,-17);
		\fill [pattern=north east lines] (\x+1,-18.4) rectangle (\x,-17);
	}
	% south wall
	\foreach \y in {-16}{
		\draw (17,\y-1) rectangle (18.4,\y);
		\fill [pattern=north east lines] (17,\y-1) rectangle (18.4,\y);
	}
	% west wall
	\foreach \x in {0, 1,..., 6}{
		\draw (\x+1,+1.4) rectangle (\x,0);
		\fill [pattern=north east lines] (\x+1,+1.4) rectangle (\x,0);
	}
	% north wall
	\foreach \y in {-0,-1,...,-16}{
		\draw (-1.4,\y-1) rectangle (0,\y);
		\fill [pattern=north east lines] (-1.4,\y-1) rectangle (0,\y);
	}
	% bottom layer
	\foreach \x in {4,6,...,18}{
		\draw (\x,-8.5) rectangle (\x+2,-10);
		\fill [pattern=north east lines] (\x,-8.5) rectangle (\x+2,-9);
		\draw (\x,-9) rectangle (\x+2,-9);
	}
	% top layer
	\foreach \x in {4,6,...,10,14}{
		\draw (\x,-7) rectangle (\x+2,-8.5);
		\fill [pattern=north east lines] (\x,-7) rectangle (\x+2,-7.5);
		\draw (\x,-7.5) rectangle (\x+2,-7.5);
	}
	% dora
	\draw (12,-7) rectangle (14,-8.5);
	\fill [pattern=north east lines] (12,-8) rectangle (14,-8.5);
	\draw (12,-8) -- (14,-8);
	% flipping the dora
	\draw [->,shorten >=1pt] (12.5,-7.75) |- (13,-6) -| (13.5,-7);
	% % dropping the tile
	\draw [dashed, thin] (16.1,-7.1) rectangle (17.9,-8.4);
	\draw [->,shorten >=1pt] (17,-7.75) -| (19,-8.5);

	% breakout lines
	\draw [|-|] (-0.3, 1.7) -- (7.3, 1.7);
	\draw [->] (3.5, 1.7) -- (3.5,3) -- (12,3) -- (12,-5.5);
	\draw [|-|] (3.7, -5.5) -- (20.3, -5.5);
\end{tikzpicture}
\caption{A diagram showing what needs to be done to properly set up the dead wall.}
\end{center}
\end{figure}
